% Options for packages loaded elsewhere
\PassOptionsToPackage{unicode}{hyperref}
\PassOptionsToPackage{hyphens}{url}
%
\documentclass[
]{article}
\usepackage{amsmath,amssymb}
\usepackage{lmodern}
\usepackage{iftex}
\ifPDFTeX
  \usepackage[T1]{fontenc}
  \usepackage[utf8]{inputenc}
  \usepackage{textcomp} % provide euro and other symbols
\else % if luatex or xetex
  \usepackage{unicode-math}
  \defaultfontfeatures{Scale=MatchLowercase}
  \defaultfontfeatures[\rmfamily]{Ligatures=TeX,Scale=1}
\fi
% Use upquote if available, for straight quotes in verbatim environments
\IfFileExists{upquote.sty}{\usepackage{upquote}}{}
\IfFileExists{microtype.sty}{% use microtype if available
  \usepackage[]{microtype}
  \UseMicrotypeSet[protrusion]{basicmath} % disable protrusion for tt fonts
}{}
\makeatletter
\@ifundefined{KOMAClassName}{% if non-KOMA class
  \IfFileExists{parskip.sty}{%
    \usepackage{parskip}
  }{% else
    \setlength{\parindent}{0pt}
    \setlength{\parskip}{6pt plus 2pt minus 1pt}}
}{% if KOMA class
  \KOMAoptions{parskip=half}}
\makeatother
\usepackage{xcolor}
\usepackage[margin=1in]{geometry}
\usepackage{color}
\usepackage{fancyvrb}
\newcommand{\VerbBar}{|}
\newcommand{\VERB}{\Verb[commandchars=\\\{\}]}
\DefineVerbatimEnvironment{Highlighting}{Verbatim}{commandchars=\\\{\}}
% Add ',fontsize=\small' for more characters per line
\usepackage{framed}
\definecolor{shadecolor}{RGB}{248,248,248}
\newenvironment{Shaded}{\begin{snugshade}}{\end{snugshade}}
\newcommand{\AlertTok}[1]{\textcolor[rgb]{0.94,0.16,0.16}{#1}}
\newcommand{\AnnotationTok}[1]{\textcolor[rgb]{0.56,0.35,0.01}{\textbf{\textit{#1}}}}
\newcommand{\AttributeTok}[1]{\textcolor[rgb]{0.77,0.63,0.00}{#1}}
\newcommand{\BaseNTok}[1]{\textcolor[rgb]{0.00,0.00,0.81}{#1}}
\newcommand{\BuiltInTok}[1]{#1}
\newcommand{\CharTok}[1]{\textcolor[rgb]{0.31,0.60,0.02}{#1}}
\newcommand{\CommentTok}[1]{\textcolor[rgb]{0.56,0.35,0.01}{\textit{#1}}}
\newcommand{\CommentVarTok}[1]{\textcolor[rgb]{0.56,0.35,0.01}{\textbf{\textit{#1}}}}
\newcommand{\ConstantTok}[1]{\textcolor[rgb]{0.00,0.00,0.00}{#1}}
\newcommand{\ControlFlowTok}[1]{\textcolor[rgb]{0.13,0.29,0.53}{\textbf{#1}}}
\newcommand{\DataTypeTok}[1]{\textcolor[rgb]{0.13,0.29,0.53}{#1}}
\newcommand{\DecValTok}[1]{\textcolor[rgb]{0.00,0.00,0.81}{#1}}
\newcommand{\DocumentationTok}[1]{\textcolor[rgb]{0.56,0.35,0.01}{\textbf{\textit{#1}}}}
\newcommand{\ErrorTok}[1]{\textcolor[rgb]{0.64,0.00,0.00}{\textbf{#1}}}
\newcommand{\ExtensionTok}[1]{#1}
\newcommand{\FloatTok}[1]{\textcolor[rgb]{0.00,0.00,0.81}{#1}}
\newcommand{\FunctionTok}[1]{\textcolor[rgb]{0.00,0.00,0.00}{#1}}
\newcommand{\ImportTok}[1]{#1}
\newcommand{\InformationTok}[1]{\textcolor[rgb]{0.56,0.35,0.01}{\textbf{\textit{#1}}}}
\newcommand{\KeywordTok}[1]{\textcolor[rgb]{0.13,0.29,0.53}{\textbf{#1}}}
\newcommand{\NormalTok}[1]{#1}
\newcommand{\OperatorTok}[1]{\textcolor[rgb]{0.81,0.36,0.00}{\textbf{#1}}}
\newcommand{\OtherTok}[1]{\textcolor[rgb]{0.56,0.35,0.01}{#1}}
\newcommand{\PreprocessorTok}[1]{\textcolor[rgb]{0.56,0.35,0.01}{\textit{#1}}}
\newcommand{\RegionMarkerTok}[1]{#1}
\newcommand{\SpecialCharTok}[1]{\textcolor[rgb]{0.00,0.00,0.00}{#1}}
\newcommand{\SpecialStringTok}[1]{\textcolor[rgb]{0.31,0.60,0.02}{#1}}
\newcommand{\StringTok}[1]{\textcolor[rgb]{0.31,0.60,0.02}{#1}}
\newcommand{\VariableTok}[1]{\textcolor[rgb]{0.00,0.00,0.00}{#1}}
\newcommand{\VerbatimStringTok}[1]{\textcolor[rgb]{0.31,0.60,0.02}{#1}}
\newcommand{\WarningTok}[1]{\textcolor[rgb]{0.56,0.35,0.01}{\textbf{\textit{#1}}}}
\usepackage{graphicx}
\makeatletter
\def\maxwidth{\ifdim\Gin@nat@width>\linewidth\linewidth\else\Gin@nat@width\fi}
\def\maxheight{\ifdim\Gin@nat@height>\textheight\textheight\else\Gin@nat@height\fi}
\makeatother
% Scale images if necessary, so that they will not overflow the page
% margins by default, and it is still possible to overwrite the defaults
% using explicit options in \includegraphics[width, height, ...]{}
\setkeys{Gin}{width=\maxwidth,height=\maxheight,keepaspectratio}
% Set default figure placement to htbp
\makeatletter
\def\fps@figure{htbp}
\makeatother
\setlength{\emergencystretch}{3em} % prevent overfull lines
\providecommand{\tightlist}{%
  \setlength{\itemsep}{0pt}\setlength{\parskip}{0pt}}
\setcounter{secnumdepth}{-\maxdimen} % remove section numbering
\ifLuaTeX
  \usepackage{selnolig}  % disable illegal ligatures
\fi
\IfFileExists{bookmark.sty}{\usepackage{bookmark}}{\usepackage{hyperref}}
\IfFileExists{xurl.sty}{\usepackage{xurl}}{} % add URL line breaks if available
\urlstyle{same} % disable monospaced font for URLs
\hypersetup{
  hidelinks,
  pdfcreator={LaTeX via pandoc}}

\author{}
\date{\vspace{-2.5em}}

\begin{document}

\begin{center}\rule{0.5\linewidth}{0.5pt}\end{center}

\hypertarget{proteome-wide-screen-for-rna-dependent-proteins-non-synchronized-a549-cells}{%
\section{\texorpdfstring{Proteome-wide Screen for RNA-dependent
Proteins: \emph{non-synchronized A549
cells}}{Proteome-wide Screen for RNA-dependent Proteins: non-synchronized A549 cells}}\label{proteome-wide-screen-for-rna-dependent-proteins-non-synchronized-a549-cells}}

Zum PDF exportieren

\begin{Shaded}
\begin{Highlighting}[]
\CommentTok{\#tinytex::install\_tinytex()}
\end{Highlighting}
\end{Shaded}

Loading the data:

\begin{Shaded}
\begin{Highlighting}[]
\NormalTok{MS\_Table }\OtherTok{\textless{}{-}} \FunctionTok{read.delim}\NormalTok{(}\StringTok{\textquotesingle{}https://www.dropbox.com/s/vm3lxljjm9chau8/RDeeP\_A549\_NS.csv?dl=1\textquotesingle{}}\NormalTok{, }\AttributeTok{header=}\ConstantTok{TRUE}\NormalTok{, }\AttributeTok{row.names=}\DecValTok{1}\NormalTok{, }\AttributeTok{sep =} \StringTok{";"}\NormalTok{)}
\end{Highlighting}
\end{Shaded}

\hypertarget{preparing-data-for-analysis}{%
\subsection{1. Preparing data for
analysis}\label{preparing-data-for-analysis}}

\hypertarget{check-for-missing-values}{%
\subsubsection{1.1. Check for missing
values}\label{check-for-missing-values}}

\begin{Shaded}
\begin{Highlighting}[]
\FunctionTok{sum}\NormalTok{(}\FunctionTok{apply}\NormalTok{(MS\_Table, }\DecValTok{1}\NormalTok{, anyNA)) }\SpecialCharTok{==} \DecValTok{0}
\end{Highlighting}
\end{Shaded}

\begin{verbatim}
## [1] TRUE
\end{verbatim}

\begin{Shaded}
\begin{Highlighting}[]
\FunctionTok{sum}\NormalTok{(}\FunctionTok{is.na}\NormalTok{(MS\_Table)) }\SpecialCharTok{==} \DecValTok{0}
\end{Highlighting}
\end{Shaded}

\begin{verbatim}
## [1] TRUE
\end{verbatim}

\hypertarget{check-data-format}{%
\subsubsection{1.2. Check data format}\label{check-data-format}}

\begin{Shaded}
\begin{Highlighting}[]
\FunctionTok{sum}\NormalTok{(}\FunctionTok{apply}\NormalTok{(MS\_Table, }\DecValTok{1}\NormalTok{, is.numeric)) }\SpecialCharTok{==} \FunctionTok{nrow}\NormalTok{(MS\_Table)}
\end{Highlighting}
\end{Shaded}

\begin{verbatim}
## [1] TRUE
\end{verbatim}

\hypertarget{deleting-rows-with-only-zeros}{%
\subsubsection{1.3. Deleting rows with only
zeros}\label{deleting-rows-with-only-zeros}}

\begin{Shaded}
\begin{Highlighting}[]
\FunctionTok{min}\NormalTok{(MS\_Table)}
\end{Highlighting}
\end{Shaded}

\begin{verbatim}
## [1] 0
\end{verbatim}

\begin{Shaded}
\begin{Highlighting}[]
\FunctionTok{sum}\NormalTok{(}\FunctionTok{apply}\NormalTok{(MS\_Table,}\DecValTok{1}\NormalTok{,sum)}\SpecialCharTok{==}\DecValTok{0}\NormalTok{)}
\end{Highlighting}
\end{Shaded}

\begin{verbatim}
## [1] 0
\end{verbatim}

-\textgreater{} da die Summe der Zeileneinträge keines Proteins 0
entspricht, wurde ein Dataframe aus False erstellt. Einträge
ausschließlich False, werden durch die sum Funktion als 0 aufaddiert.

\hypertarget{rearranging-of-data}{%
\subsubsection{1.4. Rearranging of Data}\label{rearranging-of-data}}

\hypertarget{reordering-columns}{%
\paragraph{1.4.1. Reordering columns}\label{reordering-columns}}

\begin{Shaded}
\begin{Highlighting}[]
\NormalTok{MS\_Table\_reordered }\OtherTok{\textless{}{-}}\NormalTok{ MS\_Table[, }\FunctionTok{c}\NormalTok{(}
  \FunctionTok{paste0}\NormalTok{(}\StringTok{"Fraction"}\NormalTok{, }\DecValTok{1}\SpecialCharTok{:}\DecValTok{25}\NormalTok{, }\StringTok{"\_Ctrl\_Rep1"}\NormalTok{),}
  \FunctionTok{paste0}\NormalTok{(}\StringTok{"Fraction"}\NormalTok{, }\DecValTok{1}\SpecialCharTok{:}\DecValTok{25}\NormalTok{, }\StringTok{"\_Ctrl\_Rep2"}\NormalTok{),}
  \FunctionTok{paste0}\NormalTok{(}\StringTok{"Fraction"}\NormalTok{, }\DecValTok{1}\SpecialCharTok{:}\DecValTok{25}\NormalTok{, }\StringTok{"\_Ctrl\_Rep3"}\NormalTok{),}
  \FunctionTok{paste0}\NormalTok{(}\StringTok{"Fraction"}\NormalTok{, }\DecValTok{1}\SpecialCharTok{:}\DecValTok{25}\NormalTok{, }\StringTok{"\_RNase\_Rep1"}\NormalTok{),}
  \FunctionTok{paste0}\NormalTok{(}\StringTok{"Fraction"}\NormalTok{, }\DecValTok{1}\SpecialCharTok{:}\DecValTok{25}\NormalTok{, }\StringTok{"\_RNase\_Rep2"}\NormalTok{),}
  \FunctionTok{paste0}\NormalTok{(}\StringTok{"Fraction"}\NormalTok{, }\DecValTok{1}\SpecialCharTok{:}\DecValTok{25}\NormalTok{, }\StringTok{"\_RNase\_Rep3"}\NormalTok{)}
\NormalTok{)]}
\CommentTok{\# View(MS\_Table\_reordered)}
\FunctionTok{sum}\NormalTok{(}\FunctionTok{apply}\NormalTok{(MS\_Table\_reordered, }\DecValTok{2}\NormalTok{, is.numeric)) }\SpecialCharTok{==} \FunctionTok{ncol}\NormalTok{(MS\_Table)}
\end{Highlighting}
\end{Shaded}

\begin{verbatim}
## [1] TRUE
\end{verbatim}

\hypertarget{separate-ctrl-and-rnase}{%
\paragraph{1.4.2. Separate Ctrl and
RNase}\label{separate-ctrl-and-rnase}}

\begin{Shaded}
\begin{Highlighting}[]
\NormalTok{MS\_Table\_Ctrl }\OtherTok{\textless{}{-}}\NormalTok{MS\_Table\_reordered[,}\DecValTok{1}\SpecialCharTok{:}\DecValTok{75}\NormalTok{]}
\CommentTok{\#View(MS\_Table\_Ctrl)}
\NormalTok{MS\_Table\_RNase }\OtherTok{\textless{}{-}}\NormalTok{MS\_Table\_reordered[,}\DecValTok{76}\SpecialCharTok{:}\DecValTok{150}\NormalTok{]}
\FunctionTok{View}\NormalTok{(MS\_Table\_RNase)}
\end{Highlighting}
\end{Shaded}

\hypertarget{reproducibility}{%
\subsection{2. Reproducibility}\label{reproducibility}}

Here we test whether the replicates are similar to each other. This
would mean, that the experiment is reproducible, thus the data is
reliable. Proteins that do not satisfy this condition will be removed
from the dataset and will not be analysed.

\hypertarget{pearson-correlation}{%
\paragraph{2.1 Pearson Correlation}\label{pearson-correlation}}

To facilitate the calculation of the correlation between each replicate,
we design 6 separate data frames, one for each replicate

\begin{Shaded}
\begin{Highlighting}[]
\NormalTok{ctrl.rep1.reprod }\OtherTok{\textless{}{-}}\NormalTok{ MS\_Table\_reordered[,}\DecValTok{1}\SpecialCharTok{:}\DecValTok{25}\NormalTok{]}
\NormalTok{ctrl.rep2.reprod }\OtherTok{\textless{}{-}}\NormalTok{ MS\_Table\_reordered[,}\DecValTok{26}\SpecialCharTok{:}\DecValTok{50}\NormalTok{]}
\NormalTok{ctrl.rep3.reprod }\OtherTok{\textless{}{-}}\NormalTok{ MS\_Table\_reordered[,}\DecValTok{51}\SpecialCharTok{:}\DecValTok{75}\NormalTok{]}
\NormalTok{rnase.rep1.reprod }\OtherTok{\textless{}{-}}\NormalTok{ MS\_Table\_reordered[,}\DecValTok{76}\SpecialCharTok{:}\DecValTok{100}\NormalTok{]}
\NormalTok{rnase.rep2.reprod }\OtherTok{\textless{}{-}}\NormalTok{ MS\_Table\_reordered[,}\DecValTok{101}\SpecialCharTok{:}\DecValTok{125}\NormalTok{]}
\NormalTok{rnase.rep3.reprod }\OtherTok{\textless{}{-}}\NormalTok{ MS\_Table\_reordered[,}\DecValTok{126}\SpecialCharTok{:}\DecValTok{150}\NormalTok{]}
\end{Highlighting}
\end{Shaded}

Here we calculate the correlation between the replicates and put them
together in one data frame (ctrl.cor and rnase.cor) (?)

Now we eliminate proteins which have NA-correlations (this happens when
they contain replicates with only 0s). We then create new separate data
frames for each replicate.

\begin{Shaded}
\begin{Highlighting}[]
\NormalTok{total.na }\OtherTok{\textless{}{-}} \FunctionTok{which}\NormalTok{(}\FunctionTok{rowSums}\NormalTok{(}\FunctionTok{is.na}\NormalTok{(ctrl.rnase.cor)) }\SpecialCharTok{\textgreater{}} \DecValTok{0}\NormalTok{)}
\FunctionTok{length}\NormalTok{(total.na)}
\end{Highlighting}
\end{Shaded}

\begin{verbatim}
## [1] 83
\end{verbatim}

\begin{Shaded}
\begin{Highlighting}[]
\NormalTok{MS.Table.naremoved }\OtherTok{\textless{}{-}}\NormalTok{ MS\_Table\_reordered[}\SpecialCharTok{{-}}\NormalTok{total.na,]}

\NormalTok{ctrl.rep1.naremoved }\OtherTok{\textless{}{-}}\NormalTok{ MS.Table.naremoved[,}\DecValTok{1}\SpecialCharTok{:}\DecValTok{25}\NormalTok{]}
\NormalTok{ctrl.rep2.naremoved }\OtherTok{\textless{}{-}}\NormalTok{ MS.Table.naremoved[,}\DecValTok{26}\SpecialCharTok{:}\DecValTok{50}\NormalTok{]}
\NormalTok{ctrl.rep3.naremoved }\OtherTok{\textless{}{-}}\NormalTok{ MS.Table.naremoved[,}\DecValTok{51}\SpecialCharTok{:}\DecValTok{75}\NormalTok{]}
\NormalTok{rnase.rep1.naremoved }\OtherTok{\textless{}{-}}\NormalTok{ MS.Table.naremoved[,}\DecValTok{76}\SpecialCharTok{:}\DecValTok{100}\NormalTok{]}
\NormalTok{rnase.rep2.naremoved }\OtherTok{\textless{}{-}}\NormalTok{ MS.Table.naremoved[,}\DecValTok{101}\SpecialCharTok{:}\DecValTok{125}\NormalTok{]}
\NormalTok{rnase.rep3.naremoved }\OtherTok{\textless{}{-}}\NormalTok{ MS.Table.naremoved[,}\DecValTok{126}\SpecialCharTok{:}\DecValTok{150}\NormalTok{]}
\end{Highlighting}
\end{Shaded}

Now we calculate the correlation of the replicates. This time the
proteins that containes replicates with only 0 are eliminated, so there
should be no NAs anymore.

\begin{Shaded}
\begin{Highlighting}[]
\NormalTok{ctrl.cor.naremoved }\OtherTok{\textless{}{-}} 
  \FunctionTok{cbind}\NormalTok{(ctrl.cor.rep1.rep2.naremoved }\OtherTok{\textless{}{-}} 
          \FunctionTok{sapply}\NormalTok{(}\FunctionTok{seq.int}\NormalTok{(}\FunctionTok{dim}\NormalTok{(ctrl.rep1.naremoved)[}\DecValTok{1}\NormalTok{]), }\ControlFlowTok{function}\NormalTok{(x) }\FunctionTok{cor}\NormalTok{(}\FunctionTok{as.numeric}\NormalTok{(ctrl.rep1.naremoved[x,]), }\FunctionTok{as.numeric}\NormalTok{(ctrl.rep2.naremoved[x,]))),}
\NormalTok{        ctrl.cor.rep2.rep3.naremoved }\OtherTok{\textless{}{-}} 
          \FunctionTok{sapply}\NormalTok{(}\FunctionTok{seq.int}\NormalTok{(}\FunctionTok{dim}\NormalTok{(ctrl.rep1.naremoved)[}\DecValTok{1}\NormalTok{]), }\ControlFlowTok{function}\NormalTok{(x) }\FunctionTok{cor}\NormalTok{(}\FunctionTok{as.numeric}\NormalTok{(ctrl.rep2.naremoved[x,]), }\FunctionTok{as.numeric}\NormalTok{(ctrl.rep3.naremoved[x,]))), }
\NormalTok{        ctrl.cor.rep1.rep3.naremoved }\OtherTok{\textless{}{-}} 
          \FunctionTok{sapply}\NormalTok{(}\FunctionTok{seq.int}\NormalTok{(}\FunctionTok{dim}\NormalTok{(ctrl.rep1.naremoved)[}\DecValTok{1}\NormalTok{]), }\ControlFlowTok{function}\NormalTok{(x) }\FunctionTok{cor}\NormalTok{(}\FunctionTok{as.numeric}\NormalTok{(ctrl.rep1.naremoved[x,]), }\FunctionTok{as.numeric}\NormalTok{(ctrl.rep3.naremoved[x,]))))}

\NormalTok{rnase.cor.naremoved }\OtherTok{\textless{}{-}} 
  \FunctionTok{cbind}\NormalTok{(rnase.cor.rep1.rep2.naremoved }\OtherTok{\textless{}{-}} 
         \FunctionTok{sapply}\NormalTok{(}\FunctionTok{seq.int}\NormalTok{(}\FunctionTok{dim}\NormalTok{(rnase.rep1.naremoved)[}\DecValTok{1}\NormalTok{]), }\ControlFlowTok{function}\NormalTok{(x) }\FunctionTok{cor}\NormalTok{(}\FunctionTok{as.numeric}\NormalTok{(rnase.rep1.naremoved[x,]), }\FunctionTok{as.numeric}\NormalTok{(rnase.rep2.naremoved[x,]))),}
\NormalTok{        rnase.cor.rep2.rep3.naremoved }\OtherTok{\textless{}{-}} 
          \FunctionTok{sapply}\NormalTok{(}\FunctionTok{seq.int}\NormalTok{(}\FunctionTok{dim}\NormalTok{(rnase.rep1.naremoved)[}\DecValTok{1}\NormalTok{]), }\ControlFlowTok{function}\NormalTok{(x) }\FunctionTok{cor}\NormalTok{(}\FunctionTok{as.numeric}\NormalTok{(rnase.rep2.naremoved[x,]), }\FunctionTok{as.numeric}\NormalTok{(rnase.rep3.naremoved[x,]))), }
\NormalTok{        rnase.cor.rep1.rep3.naremoved }\OtherTok{\textless{}{-}} 
          \FunctionTok{sapply}\NormalTok{(}\FunctionTok{seq.int}\NormalTok{(}\FunctionTok{dim}\NormalTok{(rnase.rep1.naremoved)[}\DecValTok{1}\NormalTok{]), }\ControlFlowTok{function}\NormalTok{(x) }\FunctionTok{cor}\NormalTok{(}\FunctionTok{as.numeric}\NormalTok{(rnase.rep1.naremoved[x,]), }\FunctionTok{as.numeric}\NormalTok{(rnase.rep3.naremoved[x,]))))}

\CommentTok{\#View(ctrl.cor.naremoved)}
\end{Highlighting}
\end{Shaded}

The following plot shows us the general distribution of correlation.

In total we look at 3*(3680-83) correlations This has to be taken into
account, when looking at the graphs. It is import to figure out if the 3
cor are for one protein or for 3 different ones.

\begin{Shaded}
\begin{Highlighting}[]
\FunctionTok{library}\NormalTok{(ggplot2)}
\end{Highlighting}
\end{Shaded}

\begin{verbatim}
## Warning: Paket 'ggplot2' wurde unter R Version 4.2.3 erstellt
\end{verbatim}

\begin{Shaded}
\begin{Highlighting}[]
\NormalTok{ctrl.cor.data.frame.naremoved }\OtherTok{\textless{}{-}} \FunctionTok{data.frame}\NormalTok{(}\FunctionTok{c}\NormalTok{(ctrl.cor.naremoved[,}\DecValTok{1}\NormalTok{],ctrl.cor.naremoved[,}\DecValTok{2}\NormalTok{],ctrl.cor.naremoved[,}\DecValTok{3}\NormalTok{]))}
\FunctionTok{colnames}\NormalTok{(ctrl.cor.data.frame.naremoved) }\OtherTok{\textless{}{-}} \StringTok{"correlation"}
\FunctionTok{ggplot}\NormalTok{(ctrl.cor.data.frame.naremoved, }\FunctionTok{aes}\NormalTok{(}\AttributeTok{x=}\NormalTok{correlation)) }\SpecialCharTok{+} \FunctionTok{geom\_histogram}\NormalTok{()}
\end{Highlighting}
\end{Shaded}

\begin{verbatim}
## `stat_bin()` using `bins = 30`. Pick better value with `binwidth`.
\end{verbatim}

\includegraphics{testfile_files/figure-latex/unnamed-chunk-15-1.pdf}

We do the same for the RNase group:

\begin{Shaded}
\begin{Highlighting}[]
\FunctionTok{library}\NormalTok{(ggplot2)}
\NormalTok{rnase.cor.data.frame.naremoved }\OtherTok{\textless{}{-}} \FunctionTok{data.frame}\NormalTok{(}\FunctionTok{c}\NormalTok{(rnase.cor.naremoved[,}\DecValTok{1}\NormalTok{],rnase.cor.naremoved[,}\DecValTok{2}\NormalTok{],rnase.cor.naremoved[,}\DecValTok{3}\NormalTok{]))}
\FunctionTok{colnames}\NormalTok{(rnase.cor.data.frame.naremoved) }\OtherTok{=} \StringTok{"correlation"}
\FunctionTok{ggplot}\NormalTok{(rnase.cor.data.frame.naremoved, }\FunctionTok{aes}\NormalTok{(}\AttributeTok{x=}\NormalTok{correlation)) }\SpecialCharTok{+} \FunctionTok{geom\_histogram}\NormalTok{()}
\end{Highlighting}
\end{Shaded}

\begin{verbatim}
## `stat_bin()` using `bins = 30`. Pick better value with `binwidth`.
\end{verbatim}

\includegraphics{testfile_files/figure-latex/unnamed-chunk-16-1.pdf}

Now we select the proteins which have correlations beneath 0.9. Those
are not reproducible, thus the data is not safe enough to be used
further.

First we determine the proteins that only have correlations under 0.9.

\begin{Shaded}
\begin{Highlighting}[]
\NormalTok{non.reproducible.ctrl }\OtherTok{\textless{}{-}} \FunctionTok{which}\NormalTok{(}\FunctionTok{rowSums}\NormalTok{(ctrl.cor.naremoved}\SpecialCharTok{\textless{}}\FloatTok{0.9}\NormalTok{)}\SpecialCharTok{\textgreater{}}\DecValTok{2}\NormalTok{)}
\CommentTok{\#length(non.reproducible.ctrl)}

\NormalTok{non.reproducible.rnase }\OtherTok{\textless{}{-}} \FunctionTok{which}\NormalTok{(}\FunctionTok{rowSums}\NormalTok{(rnase.cor.naremoved}\SpecialCharTok{\textless{}}\FloatTok{0.9}\NormalTok{)}\SpecialCharTok{\textgreater{}}\DecValTok{2}\NormalTok{)}
\CommentTok{\#length(non.reproducible.rnase)}

\NormalTok{non.reproducible }\OtherTok{\textless{}{-}} \FunctionTok{unique}\NormalTok{(}\FunctionTok{c}\NormalTok{(non.reproducible.ctrl, non.reproducible.rnase))}
\CommentTok{\#length(non.reproducible)}

\FunctionTok{length}\NormalTok{(non.reproducible.rnase)}
\end{Highlighting}
\end{Shaded}

\begin{verbatim}
## [1] 330
\end{verbatim}

\begin{Shaded}
\begin{Highlighting}[]
\FunctionTok{length}\NormalTok{(non.reproducible.ctrl)}
\end{Highlighting}
\end{Shaded}

\begin{verbatim}
## [1] 254
\end{verbatim}

\begin{Shaded}
\begin{Highlighting}[]
\CommentTok{\#View(non.reproducible2)}
\end{Highlighting}
\end{Shaded}

Now we eliminate the proteins that only have correlations under 0.9.

\begin{Shaded}
\begin{Highlighting}[]
\NormalTok{ctrl.rep }\OtherTok{\textless{}{-}}\NormalTok{ MS.Table.naremoved[}\SpecialCharTok{{-}}\NormalTok{non.reproducible,}\DecValTok{1}\SpecialCharTok{:}\DecValTok{75}\NormalTok{]}
\NormalTok{rnase.rep }\OtherTok{\textless{}{-}}\NormalTok{ MS.Table.naremoved[}\SpecialCharTok{{-}}\NormalTok{non.reproducible,}\DecValTok{76}\SpecialCharTok{:}\DecValTok{150}\NormalTok{]}

\NormalTok{ctrl.cor.removed }\OtherTok{\textless{}{-}}\NormalTok{ ctrl.cor.naremoved [}\SpecialCharTok{{-}}\NormalTok{non.reproducible,]}
\NormalTok{rnase.cor.removed }\OtherTok{\textless{}{-}}\NormalTok{ rnase.cor.naremoved [}\SpecialCharTok{{-}}\NormalTok{non.reproducible,]}


\FunctionTok{length}\NormalTok{(non.reproducible)}
\end{Highlighting}
\end{Shaded}

\begin{verbatim}
## [1] 523
\end{verbatim}

\begin{Shaded}
\begin{Highlighting}[]
\CommentTok{\#View(ctrl.rep)}
\CommentTok{\#View(MS.Table.naremoved[,1:75])}
\CommentTok{\#View(non.reproducible)}
\end{Highlighting}
\end{Shaded}

Other proteins are a bit trickier. Some proteins have two replicates
similar to each other (correlation \textless{} 0.9) and a third one that
completely differs. These Proteins have one very high and two smaller
correlations. The different replicate is often the third one (mabye
batch effect). To avoid loosing too many proteins and still to still
have safe data, we try to ignore the bad replicates. For this we first
set them to NA: After the normalization-set we can ignore them.

\begin{Shaded}
\begin{Highlighting}[]
\ControlFlowTok{for}\NormalTok{ (x }\ControlFlowTok{in} \DecValTok{1}\SpecialCharTok{:}\FunctionTok{dim}\NormalTok{(ctrl.rep)[}\DecValTok{1}\NormalTok{])\{}
  \ControlFlowTok{if}\NormalTok{ (ctrl.cor.removed[x, }\DecValTok{1}\NormalTok{] }\SpecialCharTok{\textless{}} \FloatTok{0.9}\NormalTok{) \{}
    \ControlFlowTok{if}\NormalTok{ (ctrl.cor.removed[x, }\DecValTok{3}\NormalTok{] }\SpecialCharTok{\textless{}} \FloatTok{0.9}\NormalTok{)\{}
\NormalTok{      ctrl.rep[x, }\DecValTok{1}\SpecialCharTok{:}\DecValTok{25}\NormalTok{] }\OtherTok{\textless{}{-}} \ConstantTok{NA} 
\NormalTok{    \}}
    \ControlFlowTok{if}\NormalTok{ (ctrl.cor.removed[x, }\DecValTok{2}\NormalTok{] }\SpecialCharTok{\textless{}} \FloatTok{0.9}\NormalTok{)\{}
\NormalTok{      ctrl.rep[x, }\DecValTok{26}\SpecialCharTok{:}\DecValTok{50}\NormalTok{] }\OtherTok{\textless{}{-}} \ConstantTok{NA}
\NormalTok{    \}\}}
  
  \ControlFlowTok{if}\NormalTok{ (ctrl.cor.removed[x, }\DecValTok{3}\NormalTok{] }\SpecialCharTok{\textless{}} \FloatTok{0.9}\NormalTok{) \{}
    \ControlFlowTok{if}\NormalTok{ (ctrl.cor.removed[x, }\DecValTok{2}\NormalTok{] }\SpecialCharTok{\textless{}} \FloatTok{0.9}\NormalTok{)\{}
\NormalTok{      ctrl.rep[x, }\DecValTok{51}\SpecialCharTok{:}\DecValTok{75}\NormalTok{] }\OtherTok{\textless{}{-}} \ConstantTok{NA}
\NormalTok{    \}}
\NormalTok{  \}}
\NormalTok{     \}}

\ControlFlowTok{for}\NormalTok{ (x }\ControlFlowTok{in} \DecValTok{1}\SpecialCharTok{:}\FunctionTok{dim}\NormalTok{(rnase.rep)[}\DecValTok{1}\NormalTok{])\{}
  \ControlFlowTok{if}\NormalTok{ (rnase.cor.removed[x, }\DecValTok{1}\NormalTok{] }\SpecialCharTok{\textless{}} \FloatTok{0.9}\NormalTok{) \{}
    \ControlFlowTok{if}\NormalTok{ (rnase.cor.removed[x, }\DecValTok{3}\NormalTok{] }\SpecialCharTok{\textless{}} \FloatTok{0.9}\NormalTok{)\{}
\NormalTok{      rnase.rep[x, }\DecValTok{1}\SpecialCharTok{:}\DecValTok{25}\NormalTok{] }\OtherTok{\textless{}{-}} \ConstantTok{NA} 
\NormalTok{    \}}
    \ControlFlowTok{if}\NormalTok{ (rnase.cor.removed[x, }\DecValTok{2}\NormalTok{] }\SpecialCharTok{\textless{}} \FloatTok{0.9}\NormalTok{)\{}
\NormalTok{      rnase.rep[x, }\DecValTok{26}\SpecialCharTok{:}\DecValTok{50}\NormalTok{] }\OtherTok{\textless{}{-}} \ConstantTok{NA}
\NormalTok{    \}\}}
  
  \ControlFlowTok{if}\NormalTok{ (rnase.cor.removed[x, }\DecValTok{3}\NormalTok{] }\SpecialCharTok{\textless{}} \FloatTok{0.9}\NormalTok{) \{}
    \ControlFlowTok{if}\NormalTok{ (rnase.cor.removed[x, }\DecValTok{2}\NormalTok{] }\SpecialCharTok{\textless{}} \FloatTok{0.9}\NormalTok{)\{}
\NormalTok{      rnase.rep[x, }\DecValTok{51}\SpecialCharTok{:}\DecValTok{75}\NormalTok{] }\OtherTok{\textless{}{-}} \ConstantTok{NA}
\NormalTok{    \}}
\NormalTok{  \}}
\NormalTok{\}}

\NormalTok{Nr }\OtherTok{\textless{}{-}} \FunctionTok{c}\NormalTok{(}\DecValTok{1}\SpecialCharTok{:}\FunctionTok{dim}\NormalTok{(rnase.rep)[}\DecValTok{1}\NormalTok{])}
\NormalTok{rnase.with.proteinnumbers }\OtherTok{\textless{}{-}} \FunctionTok{cbind}\NormalTok{(Nr, rnase.rep[,}\DecValTok{1}\SpecialCharTok{:}\DecValTok{25}\NormalTok{], Nr, rnase.rep[,}\DecValTok{26}\SpecialCharTok{:}\DecValTok{50}\NormalTok{], Nr, rnase.rep[,}\DecValTok{51}\SpecialCharTok{:}\DecValTok{75}\NormalTok{])}
\CommentTok{\#View(rnase.with.proteinnumbers)}
\CommentTok{\#View(rnase.cor.removed)}

\NormalTok{ctrl.with.proteinnumbers }\OtherTok{\textless{}{-}} \FunctionTok{cbind}\NormalTok{(Nr, ctrl.rep[,}\DecValTok{1}\SpecialCharTok{:}\DecValTok{25}\NormalTok{], Nr, ctrl.rep[,}\DecValTok{26}\SpecialCharTok{:}\DecValTok{50}\NormalTok{], Nr, ctrl.rep[,}\DecValTok{51}\SpecialCharTok{:}\DecValTok{75}\NormalTok{])}
\CommentTok{\#View(ctrl.with.proteinnumbers)}
\end{Highlighting}
\end{Shaded}

We now have 3074 Proteins left. They are stored in new variables:

We now have clean data, with proteins that have reproducible data we can
use for further analysis.

\hypertarget{scaled-and-reduced-dataset}{%
\subsection{3. Scaled and Reduced
Dataset}\label{scaled-and-reduced-dataset}}

For the normalization each replicate has to be separated, therefore we
design 6 separate dataframes.

\begin{Shaded}
\begin{Highlighting}[]
\NormalTok{ctrl.rep1 }\OtherTok{\textless{}{-}}\NormalTok{ ctrl.rep[,}\DecValTok{1}\SpecialCharTok{:}\DecValTok{25}\NormalTok{]}
\NormalTok{ctrl.rep2 }\OtherTok{\textless{}{-}}\NormalTok{ ctrl.rep[,}\DecValTok{26}\SpecialCharTok{:}\DecValTok{50}\NormalTok{]}
\NormalTok{ctrl.rep3 }\OtherTok{\textless{}{-}}\NormalTok{ ctrl.rep[,}\DecValTok{51}\SpecialCharTok{:}\DecValTok{75}\NormalTok{]}
\NormalTok{rnase.rep1 }\OtherTok{\textless{}{-}}\NormalTok{ rnase.rep[,}\DecValTok{1}\SpecialCharTok{:}\DecValTok{25}\NormalTok{]}
\NormalTok{rnase.rep2 }\OtherTok{\textless{}{-}}\NormalTok{ rnase.rep[,}\DecValTok{26}\SpecialCharTok{:}\DecValTok{50}\NormalTok{]}
\NormalTok{rnase.rep3 }\OtherTok{\textless{}{-}}\NormalTok{ rnase.rep[,}\DecValTok{51}\SpecialCharTok{:}\DecValTok{75}\NormalTok{]}
\end{Highlighting}
\end{Shaded}

\hypertarget{mean-value-method}{%
\subsubsection{3.1. Mean Value Method}\label{mean-value-method}}

\hypertarget{normalization}{%
\paragraph{3.1.1. Normalization}\label{normalization}}

We perform the mean-value-method (mvm) on each replicate, both control
and RNase:

\begin{Shaded}
\begin{Highlighting}[]
\CommentTok{\# Control Replicate 1 MVM}
\NormalTok{ctrl.rep1.mvm.norm }\OtherTok{\textless{}{-}} \FunctionTok{t}\NormalTok{(}\FunctionTok{apply}\NormalTok{(ctrl.rep1, }\DecValTok{1}\NormalTok{, }\ControlFlowTok{function}\NormalTok{(x) \{}
\NormalTok{  normalized }\OtherTok{\textless{}{-}}\NormalTok{ x }\SpecialCharTok{{-}} \FunctionTok{mean}\NormalTok{(x)}
\NormalTok{  normalized[normalized }\SpecialCharTok{\textless{}} \DecValTok{0}\NormalTok{] }\OtherTok{\textless{}{-}} \DecValTok{0}
\NormalTok{  scaled }\OtherTok{\textless{}{-}}\NormalTok{ normalized }\SpecialCharTok{*}\NormalTok{ (}\DecValTok{100} \SpecialCharTok{/} \FunctionTok{sum}\NormalTok{(normalized))}
  \FunctionTok{return}\NormalTok{(scaled)}
\NormalTok{\}))}

\CommentTok{\# View(ctrl.rep1.mvm.norm)}

\CommentTok{\# Control Replicate 2 MVM}
\NormalTok{ctrl.rep2.mvm.norm }\OtherTok{\textless{}{-}} \FunctionTok{t}\NormalTok{(}\FunctionTok{apply}\NormalTok{(ctrl.rep2, }\DecValTok{1}\NormalTok{, }\ControlFlowTok{function}\NormalTok{(x) \{}
\NormalTok{  normalized }\OtherTok{\textless{}{-}}\NormalTok{ x }\SpecialCharTok{{-}} \FunctionTok{mean}\NormalTok{(x)}
\NormalTok{  normalized[normalized }\SpecialCharTok{\textless{}} \DecValTok{0}\NormalTok{] }\OtherTok{\textless{}{-}} \DecValTok{0}
\NormalTok{  scaled }\OtherTok{\textless{}{-}}\NormalTok{ normalized }\SpecialCharTok{*}\NormalTok{ (}\DecValTok{100} \SpecialCharTok{/} \FunctionTok{sum}\NormalTok{(normalized))}
  \FunctionTok{return}\NormalTok{(scaled)}
\NormalTok{\}))}

\CommentTok{\# Control Replicate 3 MVM}
\NormalTok{ctrl.rep3.mvm.norm }\OtherTok{\textless{}{-}} \FunctionTok{t}\NormalTok{(}\FunctionTok{apply}\NormalTok{(ctrl.rep3, }\DecValTok{1}\NormalTok{, }\ControlFlowTok{function}\NormalTok{(x) \{}
\NormalTok{  normalized }\OtherTok{\textless{}{-}}\NormalTok{ x }\SpecialCharTok{{-}} \FunctionTok{mean}\NormalTok{(x)}
\NormalTok{  normalized[normalized }\SpecialCharTok{\textless{}} \DecValTok{0}\NormalTok{] }\OtherTok{\textless{}{-}} \DecValTok{0}
\NormalTok{  scaled }\OtherTok{\textless{}{-}}\NormalTok{ normalized }\SpecialCharTok{*}\NormalTok{ (}\DecValTok{100} \SpecialCharTok{/} \FunctionTok{sum}\NormalTok{(normalized))}
  \FunctionTok{return}\NormalTok{(scaled)}
\NormalTok{\}))}

\CommentTok{\# RNase Replicate 1 MVM}
\NormalTok{rnase.rep1.mvm.norm }\OtherTok{\textless{}{-}} \FunctionTok{t}\NormalTok{(}\FunctionTok{apply}\NormalTok{(rnase.rep1, }\DecValTok{1}\NormalTok{, }\ControlFlowTok{function}\NormalTok{(x) \{}
\NormalTok{  normalized }\OtherTok{\textless{}{-}}\NormalTok{ x }\SpecialCharTok{{-}} \FunctionTok{mean}\NormalTok{(x)}
\NormalTok{  normalized[normalized }\SpecialCharTok{\textless{}} \DecValTok{0}\NormalTok{] }\OtherTok{\textless{}{-}} \DecValTok{0}
\NormalTok{  scaled }\OtherTok{\textless{}{-}}\NormalTok{ normalized }\SpecialCharTok{*}\NormalTok{ (}\DecValTok{100} \SpecialCharTok{/} \FunctionTok{sum}\NormalTok{(normalized))}
  \FunctionTok{return}\NormalTok{(scaled)}
\NormalTok{\}))}

\CommentTok{\# RNase Replicate 2 MVM}
\NormalTok{rnase.rep2.mvm.norm }\OtherTok{\textless{}{-}} \FunctionTok{t}\NormalTok{(}\FunctionTok{apply}\NormalTok{(rnase.rep2, }\DecValTok{1}\NormalTok{, }\ControlFlowTok{function}\NormalTok{(x) \{}
\NormalTok{  normalized }\OtherTok{\textless{}{-}}\NormalTok{ x }\SpecialCharTok{{-}} \FunctionTok{mean}\NormalTok{(x)}
\NormalTok{  normalized[normalized }\SpecialCharTok{\textless{}} \DecValTok{0}\NormalTok{] }\OtherTok{\textless{}{-}} \DecValTok{0}
\NormalTok{  scaled }\OtherTok{\textless{}{-}}\NormalTok{ normalized }\SpecialCharTok{*}\NormalTok{ (}\DecValTok{100} \SpecialCharTok{/} \FunctionTok{sum}\NormalTok{(normalized))}
  \FunctionTok{return}\NormalTok{(scaled)}
\NormalTok{\}))}

\CommentTok{\# RNase Replicate 3 MVM}
\NormalTok{rnase.rep3.mvm.norm }\OtherTok{\textless{}{-}} \FunctionTok{t}\NormalTok{(}\FunctionTok{apply}\NormalTok{(rnase.rep3, }\DecValTok{1}\NormalTok{, }\ControlFlowTok{function}\NormalTok{(x) \{}
\NormalTok{  normalized }\OtherTok{\textless{}{-}}\NormalTok{ x }\SpecialCharTok{{-}} \FunctionTok{mean}\NormalTok{(x)}
\NormalTok{  normalized[normalized }\SpecialCharTok{\textless{}} \DecValTok{0}\NormalTok{] }\OtherTok{\textless{}{-}} \DecValTok{0}
\NormalTok{  scaled }\OtherTok{\textless{}{-}}\NormalTok{ normalized }\SpecialCharTok{*}\NormalTok{ (}\DecValTok{100} \SpecialCharTok{/} \FunctionTok{sum}\NormalTok{(normalized))}
  \FunctionTok{return}\NormalTok{(scaled)}
\NormalTok{\}))}
\end{Highlighting}
\end{Shaded}

\hypertarget{reduction}{%
\paragraph{3.1.2. Reduction}\label{reduction}}

To reduce we take the mean value between each replicate. Here we must
consider the NA-values of non-reproducible replicates.

\begin{Shaded}
\begin{Highlighting}[]
\NormalTok{r1c }\OtherTok{\textless{}{-}}\NormalTok{ ctrl.rep1.mvm.norm}
\NormalTok{r2c }\OtherTok{\textless{}{-}}\NormalTok{ ctrl.rep2.mvm.norm}
\NormalTok{r3c }\OtherTok{\textless{}{-}}\NormalTok{ ctrl.rep3.mvm.norm}

\NormalTok{r1c0 }\OtherTok{\textless{}{-}}\NormalTok{ ctrl.rep1.mvm.norm}
\NormalTok{r2c0 }\OtherTok{\textless{}{-}}\NormalTok{ ctrl.rep2.mvm.norm}
\NormalTok{r3c0 }\OtherTok{\textless{}{-}}\NormalTok{ ctrl.rep3.mvm.norm}

\NormalTok{r1c0[}\FunctionTok{is.na}\NormalTok{(r1c)] }\OtherTok{\textless{}{-}} \DecValTok{0}
\NormalTok{r2c0[}\FunctionTok{is.na}\NormalTok{(r2c)] }\OtherTok{\textless{}{-}} \DecValTok{0}
\NormalTok{r3c0[}\FunctionTok{is.na}\NormalTok{(r3c)] }\OtherTok{\textless{}{-}} \DecValTok{0}


\NormalTok{ctrl.mvm.reduced }\OtherTok{\textless{}{-}}\NormalTok{  (r1c0 }\SpecialCharTok{+}\NormalTok{ r2c0 }\SpecialCharTok{+}\NormalTok{ r3c0)}\SpecialCharTok{/}\NormalTok{(}\DecValTok{3}\SpecialCharTok{{-}}\NormalTok{((}\FunctionTok{sum}\NormalTok{(}\FunctionTok{is.na}\NormalTok{(r1c[x,]) }\SpecialCharTok{+} \FunctionTok{is.na}\NormalTok{(r2c[x,]) }\SpecialCharTok{+} \FunctionTok{is.na}\NormalTok{(r3c[x,])))}\SpecialCharTok{/}\DecValTok{25}\NormalTok{))}
\end{Highlighting}
\end{Shaded}

\begin{Shaded}
\begin{Highlighting}[]
\NormalTok{r1r }\OtherTok{\textless{}{-}}\NormalTok{ rnase.rep1.mvm.norm}
\NormalTok{r2r }\OtherTok{\textless{}{-}}\NormalTok{ rnase.rep2.mvm.norm}
\NormalTok{r3r }\OtherTok{\textless{}{-}}\NormalTok{ rnase.rep3.mvm.norm}

\NormalTok{r1r0 }\OtherTok{\textless{}{-}}\NormalTok{ rnase.rep1.mvm.norm}
\NormalTok{r2r0 }\OtherTok{\textless{}{-}}\NormalTok{ rnase.rep2.mvm.norm}
\NormalTok{r3r0 }\OtherTok{\textless{}{-}}\NormalTok{ rnase.rep3.mvm.norm}

\NormalTok{r1r0[}\FunctionTok{is.na}\NormalTok{(r1r)] }\OtherTok{\textless{}{-}} \DecValTok{0}
\NormalTok{r2r0[}\FunctionTok{is.na}\NormalTok{(r2r)] }\OtherTok{\textless{}{-}} \DecValTok{0}
\NormalTok{r3r0[}\FunctionTok{is.na}\NormalTok{(r3r)] }\OtherTok{\textless{}{-}} \DecValTok{0}

\NormalTok{rnase.mvm.reduced }\OtherTok{\textless{}{-}}\NormalTok{  (r1r0 }\SpecialCharTok{+}\NormalTok{ r2r0 }\SpecialCharTok{+}\NormalTok{ r3r0)}\SpecialCharTok{/}\NormalTok{(}\DecValTok{3} \SpecialCharTok{{-}}\NormalTok{ ((}\FunctionTok{sum}\NormalTok{(}\FunctionTok{is.na}\NormalTok{(r1r[x,]) }\SpecialCharTok{+} \FunctionTok{is.na}\NormalTok{(r2r[x,]) }\SpecialCharTok{+} \FunctionTok{is.na}\NormalTok{(r3r[x,])))}\SpecialCharTok{/}\DecValTok{25}\NormalTok{))}
                        

\CommentTok{\#View(rnase.mvm.reduced)}
\end{Highlighting}
\end{Shaded}

\hypertarget{scaling}{%
\paragraph{3.1.3. Scaling}\label{scaling}}

To test whether we have ``lost'' our scaling during the merge, and find
out whether scaling back to 100 is necessary, we scale the control to
100 and compare it with the original control.

\begin{Shaded}
\begin{Highlighting}[]
\NormalTok{ctrl.mvm.scaled }\OtherTok{=} 
  \FunctionTok{sweep}\NormalTok{(ctrl.mvm.reduced,}\DecValTok{1}\NormalTok{,}\DecValTok{100}\SpecialCharTok{/}\FunctionTok{rowSums}\NormalTok{(ctrl.mvm.reduced),}\StringTok{\textquotesingle{}*\textquotesingle{}}\NormalTok{)}

\CommentTok{\# Check if the two data frames are identical}
\NormalTok{is\_identical }\OtherTok{\textless{}{-}} \FunctionTok{identical}\NormalTok{(ctrl.mvm.reduced, ctrl.mvm.scaled)}

\CommentTok{\# Print the result}
\ControlFlowTok{if}\NormalTok{ (is\_identical) \{}
  \FunctionTok{print}\NormalTok{(}\StringTok{"The data frames are identical."}\NormalTok{)}
\NormalTok{ \} }\ControlFlowTok{else}\NormalTok{ \{}
  \FunctionTok{print}\NormalTok{(}\StringTok{"The data frames are not identical."}\NormalTok{)}
\NormalTok{ \}}
\end{Highlighting}
\end{Shaded}

\begin{verbatim}
## [1] "The data frames are not identical."
\end{verbatim}

-\textgreater{} scaling back to 100 is necessary

Because scaling back to 100 is necessary, we do it for the RNase too:

\begin{Shaded}
\begin{Highlighting}[]
\NormalTok{rnase.mvm.scaled }\OtherTok{=} 
  \FunctionTok{sweep}\NormalTok{(rnase.mvm.reduced,}\DecValTok{1}\NormalTok{,}\DecValTok{100}\SpecialCharTok{/}\FunctionTok{rowSums}\NormalTok{(rnase.mvm.reduced),}\StringTok{\textquotesingle{}*\textquotesingle{}}\NormalTok{)}
\end{Highlighting}
\end{Shaded}

Now we have normalized our data using the mean-value-method, and scaled
it to 100. The two variables that will be used later on either contain
the normalized (mvm) and scaled data of the control: \textbf{ctrl.mvm}
or the normalized (mvm) and scaled data of the rnase: \textbf{rnase.mvm}

\begin{Shaded}
\begin{Highlighting}[]
\NormalTok{new.colnames.ctrl }\OtherTok{\textless{}{-}} \FunctionTok{c}\NormalTok{(}\StringTok{"Fraction\_1\_Ctrl"}\NormalTok{,}\StringTok{"Fraction\_2\_Ctrl"}\NormalTok{,}\StringTok{"Fraction\_3\_Ctrl"}\NormalTok{,}\StringTok{"Fraction\_4\_Ctrl"}\NormalTok{,}\StringTok{"Fraction\_5\_Ctrl"}\NormalTok{,}\StringTok{"Fraction\_6\_Ctrl"}\NormalTok{,}\StringTok{"Fraction\_7\_Ctrl"}\NormalTok{,}\StringTok{"Fraction\_8\_Ctrl"}\NormalTok{,}\StringTok{"Fraction\_9\_Ctrl"}\NormalTok{,}\StringTok{"Fraction\_10\_Ctrl"}\NormalTok{,}\StringTok{"Fraction\_11\_Ctrl"}\NormalTok{,}\StringTok{"Fraction\_12\_Ctrl"}\NormalTok{,}\StringTok{"Fraction\_13\_Ctrl"}\NormalTok{,}\StringTok{"Fraction\_14\_Ctrl"}\NormalTok{,}\StringTok{"Fraction\_15\_Ctrl"}\NormalTok{,}\StringTok{"Fraction\_16\_Ctrl"}\NormalTok{,}\StringTok{"Fraction\_17\_Ctrl"}\NormalTok{,}\StringTok{"Fraction\_18\_Ctrl"}\NormalTok{,}\StringTok{"Fraction\_19\_Ctrl"}\NormalTok{,}\StringTok{"Fraction\_20\_Ctrl"}\NormalTok{,}\StringTok{"Fraction\_21\_Ctrl"}\NormalTok{,}\StringTok{"Fraction\_22\_Ctrl"}\NormalTok{,}\StringTok{"Fraction\_23\_Ctrl"}\NormalTok{,}\StringTok{"Fraction\_24\_Ctrl"}\NormalTok{,}\StringTok{"Fraction\_25\_Ctrl"}\NormalTok{)}

\NormalTok{new.colnames.rnase }\OtherTok{\textless{}{-}} \FunctionTok{c}\NormalTok{(}\StringTok{"Fraction\_1\_RNase"}\NormalTok{,}\StringTok{"Fraction\_2\_RNase"}\NormalTok{,}\StringTok{"Fraction\_3\_RNase"}\NormalTok{,}\StringTok{"Fraction\_4\_RNase"}\NormalTok{,}\StringTok{"Fraction\_5\_RNase"}\NormalTok{,}\StringTok{"Fraction\_6\_RNase"}\NormalTok{,}\StringTok{"Fraction\_7\_RNase"}\NormalTok{,}\StringTok{"Fraction\_8\_RNase"}\NormalTok{,}\StringTok{"Fraction\_9\_RNase"}\NormalTok{,}\StringTok{"Fraction\_10\_RNase"}\NormalTok{,}\StringTok{"Fraction\_11\_RNase"}\NormalTok{,}\StringTok{"Fraction\_12\_RNase"}\NormalTok{,}\StringTok{"Fraction\_13\_RNase"}\NormalTok{,}\StringTok{"Fraction\_14\_RNase"}\NormalTok{,}\StringTok{"Fraction\_15\_RNase"}\NormalTok{,}\StringTok{"Fraction\_16\_RNase"}\NormalTok{,}\StringTok{"Fraction\_17\_RNase"}\NormalTok{,}\StringTok{"Fraction\_18\_RNase"}\NormalTok{,}\StringTok{"Fraction\_19\_RNase"}\NormalTok{,}\StringTok{"Fraction\_20\_RNase"}\NormalTok{,}\StringTok{"Fraction\_21\_RNase"}\NormalTok{,}\StringTok{"Fraction\_22\_RNase"}\NormalTok{,}\StringTok{"Fraction\_23\_RNase"}\NormalTok{,}\StringTok{"Fraction\_24\_RNase"}\NormalTok{,}\StringTok{"Fraction\_25\_RNase"}\NormalTok{)}

\NormalTok{ctrl.mvm }\OtherTok{\textless{}{-}}\NormalTok{ ctrl.mvm.scaled}
\FunctionTok{colnames}\NormalTok{(ctrl.mvm) }\OtherTok{\textless{}{-}}\NormalTok{ new.colnames.ctrl}

\NormalTok{rnase.mvm }\OtherTok{\textless{}{-}}\NormalTok{ rnase.mvm.scaled}
\FunctionTok{colnames}\NormalTok{(rnase.mvm) }\OtherTok{\textless{}{-}}\NormalTok{ new.colnames.rnase}
\end{Highlighting}
\end{Shaded}

\hypertarget{z---transformation}{%
\subsubsection{3.2. z - Transformation}\label{z---transformation}}

\hypertarget{normalization-1}{%
\paragraph{3.2.1. Normalization}\label{normalization-1}}

The z-Transformation does not work with df that have NA-values. This
means we cannot use the ctrl.clean and rnase.clean df. We have to use
the old ctrl.rep or the MS.Table.naremoved{[}-non.reproducible,1:75{]}
dfs. On these we first perform z-Transformation and then we use the same
algorithms as before to reduce the dataset in regards to
reproducibility. (boah mein Elglisch)

\begin{Shaded}
\begin{Highlighting}[]
\NormalTok{ctrl.rep.zt }\OtherTok{\textless{}{-}}\NormalTok{ MS.Table.naremoved[}\SpecialCharTok{{-}}\NormalTok{non.reproducible,}\DecValTok{1}\SpecialCharTok{:}\DecValTok{75}\NormalTok{]}
\NormalTok{rnase.rep.zt }\OtherTok{\textless{}{-}}\NormalTok{ MS.Table.naremoved[}\SpecialCharTok{{-}}\NormalTok{non.reproducible,}\DecValTok{76}\SpecialCharTok{:}\DecValTok{150}\NormalTok{]}

\NormalTok{ctrl.rep1z }\OtherTok{\textless{}{-}}\NormalTok{ ctrl.rep.zt[,}\DecValTok{1}\SpecialCharTok{:}\DecValTok{25}\NormalTok{]}
\NormalTok{ctrl.rep2z }\OtherTok{\textless{}{-}}\NormalTok{ ctrl.rep.zt[,}\DecValTok{26}\SpecialCharTok{:}\DecValTok{50}\NormalTok{]}
\NormalTok{ctrl.rep3z }\OtherTok{\textless{}{-}}\NormalTok{ ctrl.rep.zt[,}\DecValTok{51}\SpecialCharTok{:}\DecValTok{75}\NormalTok{]}
\NormalTok{ctrl.repz }\OtherTok{\textless{}{-}} \FunctionTok{cbind}\NormalTok{(ctrl.rep1z,ctrl.rep2z,ctrl.rep3z)}


\NormalTok{rnase.rep1z }\OtherTok{\textless{}{-}}\NormalTok{ rnase.rep.zt[,}\DecValTok{1}\SpecialCharTok{:}\DecValTok{25}\NormalTok{]}
\NormalTok{rnase.rep2z }\OtherTok{\textless{}{-}}\NormalTok{ rnase.rep.zt[,}\DecValTok{26}\SpecialCharTok{:}\DecValTok{50}\NormalTok{]}
\NormalTok{rnase.rep3z }\OtherTok{\textless{}{-}}\NormalTok{ rnase.rep.zt[,}\DecValTok{51}\SpecialCharTok{:}\DecValTok{75}\NormalTok{]}
\NormalTok{rnase.repz }\OtherTok{\textless{}{-}} \FunctionTok{cbind}\NormalTok{(rnase.rep1z,rnase.rep2z,rnase.rep3z)}
\end{Highlighting}
\end{Shaded}

First the normalization for the Ctrl:~

Since the protein amount in each replicate is different, it is better to
calculate the mean for each replicate separately. However the sd-value
does not have to be adapted. Because the replicates have the same
variance (same procedure for every replicate in the wet lab), we don't
have to calculate the standard devidation for the replicates in each
fraction extra. We can calculate the sd-value for one row /sd.

\emph{sd-values and mean values of Ctrl:}

\begin{Shaded}
\begin{Highlighting}[]
\NormalTok{sd.ctrl }\OtherTok{\textless{}{-}} \FunctionTok{apply}\NormalTok{(ctrl.repz, }\DecValTok{1}\NormalTok{, sd)}

\NormalTok{mean.ctrl.rep1 }\OtherTok{\textless{}{-}} \FunctionTok{apply}\NormalTok{(ctrl.rep1z, }\DecValTok{1}\NormalTok{, mean)}
\NormalTok{mean.ctrl.rep2 }\OtherTok{\textless{}{-}} \FunctionTok{apply}\NormalTok{(ctrl.rep2z, }\DecValTok{1}\NormalTok{, mean)}
\NormalTok{mean.ctrl.rep3 }\OtherTok{\textless{}{-}} \FunctionTok{apply}\NormalTok{(ctrl.rep3z, }\DecValTok{1}\NormalTok{, mean)}
\end{Highlighting}
\end{Shaded}

\emph{Normalization of Ctrl:}

\begin{Shaded}
\begin{Highlighting}[]
\NormalTok{ctrl.rep1.meanvalue }\OtherTok{\textless{}{-}} \FunctionTok{sweep}\NormalTok{(ctrl.rep1z,}\DecValTok{1}\NormalTok{,mean.ctrl.rep1,}\StringTok{\textquotesingle{}{-}\textquotesingle{}}\NormalTok{)}
\NormalTok{ctrl.rep1.zt.norm }\OtherTok{\textless{}{-}} \FunctionTok{sweep}\NormalTok{(ctrl.rep1.meanvalue,}\DecValTok{1}\NormalTok{,sd.ctrl,}\StringTok{\textquotesingle{}/\textquotesingle{}}\NormalTok{)}

\NormalTok{ctrl.rep2.meanvalue }\OtherTok{\textless{}{-}} \FunctionTok{sweep}\NormalTok{(ctrl.rep2z,}\DecValTok{1}\NormalTok{,mean.ctrl.rep2,}\StringTok{\textquotesingle{}{-}\textquotesingle{}}\NormalTok{)}
\NormalTok{ctrl.rep2.zt.norm }\OtherTok{\textless{}{-}} \FunctionTok{sweep}\NormalTok{(ctrl.rep2.meanvalue,}\DecValTok{1}\NormalTok{,sd.ctrl,}\StringTok{\textquotesingle{}/\textquotesingle{}}\NormalTok{)}

\NormalTok{ctrl.rep3.meanvalue }\OtherTok{\textless{}{-}} \FunctionTok{sweep}\NormalTok{(ctrl.rep3z,}\DecValTok{1}\NormalTok{,mean.ctrl.rep3,}\StringTok{\textquotesingle{}{-}\textquotesingle{}}\NormalTok{)}
\NormalTok{ctrl.rep3.zt.norm }\OtherTok{\textless{}{-}} \FunctionTok{sweep}\NormalTok{(ctrl.rep3.meanvalue,}\DecValTok{1}\NormalTok{,sd.ctrl,}\StringTok{\textquotesingle{}/\textquotesingle{}}\NormalTok{)}

\CommentTok{\#calculate again the min value for each protein}
\NormalTok{min.ctrl.rep1.zt.norm }\OtherTok{\textless{}{-}} \FunctionTok{apply}\NormalTok{(ctrl.rep1.zt.norm, }\DecValTok{1}\NormalTok{, min)}
\NormalTok{min.ctrl.rep2.zt.norm }\OtherTok{\textless{}{-}} \FunctionTok{apply}\NormalTok{(ctrl.rep2.zt.norm, }\DecValTok{1}\NormalTok{, min)}
\NormalTok{min.ctrl.rep3.zt.norm }\OtherTok{\textless{}{-}} \FunctionTok{apply}\NormalTok{(ctrl.rep3.zt.norm, }\DecValTok{1}\NormalTok{, min)}

\CommentTok{\#substract the min value from each position to discard the 0}
\NormalTok{ctrl.rep1.zt.norm.pos }\OtherTok{\textless{}{-}} \FunctionTok{sweep}\NormalTok{(ctrl.rep1.zt.norm,}\DecValTok{1}\NormalTok{,min.ctrl.rep1.zt.norm,}\AttributeTok{FUN =} \StringTok{\textquotesingle{}{-}\textquotesingle{}}\NormalTok{)}
\NormalTok{ctrl.rep2.zt.norm.pos }\OtherTok{\textless{}{-}} \FunctionTok{sweep}\NormalTok{(ctrl.rep2.zt.norm,}\DecValTok{1}\NormalTok{,min.ctrl.rep2.zt.norm,}\AttributeTok{FUN =} \StringTok{\textquotesingle{}{-}\textquotesingle{}}\NormalTok{)}
\NormalTok{ctrl.rep3.zt.norm.pos }\OtherTok{\textless{}{-}} \FunctionTok{sweep}\NormalTok{(ctrl.rep3.zt.norm,}\DecValTok{1}\NormalTok{,min.ctrl.rep3.zt.norm,}\AttributeTok{FUN =} \StringTok{\textquotesingle{}{-}\textquotesingle{}}\NormalTok{)}
\end{Highlighting}
\end{Shaded}

\emph{sd-values and mean values of RNase:}

\begin{Shaded}
\begin{Highlighting}[]
\NormalTok{sd.rnase }\OtherTok{=} \FunctionTok{apply}\NormalTok{(rnase.repz, }\DecValTok{1}\NormalTok{, sd)}

\NormalTok{mean.rnase.rep1 }\OtherTok{\textless{}{-}} \FunctionTok{apply}\NormalTok{(rnase.rep1z, }\DecValTok{1}\NormalTok{, mean)}
\NormalTok{mean.rnase.rep2 }\OtherTok{\textless{}{-}} \FunctionTok{apply}\NormalTok{(rnase.rep2z, }\DecValTok{1}\NormalTok{, mean)}
\NormalTok{mean.rnase.rep3 }\OtherTok{\textless{}{-}} \FunctionTok{apply}\NormalTok{(rnase.rep3z, }\DecValTok{1}\NormalTok{, mean)}
\end{Highlighting}
\end{Shaded}

\emph{Normalization of RNase:}

\begin{Shaded}
\begin{Highlighting}[]
\NormalTok{rnase.rep1.meanvalue }\OtherTok{\textless{}{-}} \FunctionTok{sweep}\NormalTok{(rnase.rep1z,}\DecValTok{1}\NormalTok{,mean.rnase.rep1,}\StringTok{\textquotesingle{}{-}\textquotesingle{}}\NormalTok{)}
\NormalTok{rnase.rep1.zt.norm }\OtherTok{\textless{}{-}} \FunctionTok{sweep}\NormalTok{(rnase.rep1.meanvalue,}\DecValTok{1}\NormalTok{,sd.rnase,}\StringTok{\textquotesingle{}/\textquotesingle{}}\NormalTok{)}

\NormalTok{rnase.rep2.meanvalue }\OtherTok{\textless{}{-}} \FunctionTok{sweep}\NormalTok{(rnase.rep2z,}\DecValTok{1}\NormalTok{,mean.rnase.rep2,}\StringTok{\textquotesingle{}{-}\textquotesingle{}}\NormalTok{)}
\NormalTok{rnase.rep2.zt.norm }\OtherTok{\textless{}{-}} \FunctionTok{sweep}\NormalTok{(rnase.rep2.meanvalue,}\DecValTok{1}\NormalTok{,sd.rnase,}\StringTok{\textquotesingle{}/\textquotesingle{}}\NormalTok{)}

\NormalTok{rnase.rep3.meanvalue }\OtherTok{\textless{}{-}} \FunctionTok{sweep}\NormalTok{(rnase.rep3z,}\DecValTok{1}\NormalTok{,mean.rnase.rep3,}\StringTok{\textquotesingle{}{-}\textquotesingle{}}\NormalTok{)}
\NormalTok{rnase.rep3.zt.norm }\OtherTok{\textless{}{-}} \FunctionTok{sweep}\NormalTok{(rnase.rep3.meanvalue,}\DecValTok{1}\NormalTok{,sd.rnase,}\StringTok{\textquotesingle{}/\textquotesingle{}}\NormalTok{)}

\CommentTok{\#calculate again the min value for each protein}
\NormalTok{min.rnase.rep1.zt.norm }\OtherTok{\textless{}{-}} \FunctionTok{apply}\NormalTok{(rnase.rep1.zt.norm, }\DecValTok{1}\NormalTok{, min)}
\NormalTok{min.rnase.rep2.zt.norm }\OtherTok{\textless{}{-}} \FunctionTok{apply}\NormalTok{(rnase.rep2.zt.norm, }\DecValTok{1}\NormalTok{, min)}
\NormalTok{min.rnase.rep3.zt.norm }\OtherTok{\textless{}{-}} \FunctionTok{apply}\NormalTok{(rnase.rep3.zt.norm, }\DecValTok{1}\NormalTok{, min)}

\CommentTok{\#substract the min value from each position to discard the 0}
\NormalTok{rnase.rep1.zt.norm.pos }\OtherTok{\textless{}{-}} \FunctionTok{sweep}\NormalTok{(rnase.rep1.zt.norm,}\DecValTok{1}\NormalTok{,min.rnase.rep1.zt.norm,}\AttributeTok{FUN =} \StringTok{\textquotesingle{}{-}\textquotesingle{}}\NormalTok{)}
\NormalTok{rnase.rep2.zt.norm.pos }\OtherTok{\textless{}{-}} \FunctionTok{sweep}\NormalTok{(rnase.rep2.zt.norm,}\DecValTok{1}\NormalTok{,min.rnase.rep2.zt.norm,}\AttributeTok{FUN =} \StringTok{\textquotesingle{}{-}\textquotesingle{}}\NormalTok{)}
\NormalTok{rnase.rep3.zt.norm.pos }\OtherTok{\textless{}{-}} \FunctionTok{sweep}\NormalTok{(rnase.rep3.zt.norm,}\DecValTok{1}\NormalTok{,min.rnase.rep3.zt.norm,}\AttributeTok{FUN =} \StringTok{\textquotesingle{}{-}\textquotesingle{}}\NormalTok{)}
\end{Highlighting}
\end{Shaded}


\end{document}
